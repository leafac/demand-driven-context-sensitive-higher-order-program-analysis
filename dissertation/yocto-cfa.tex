% !TEX program = lualatex
% !TEX options = -shell-escape -synctex=1 -interaction=nonstopmode -file-line-error "%DOC%"
% !BIB program = bibtex

\documentclass[12pt, oneside]{book}

\usepackage[a-1b]{pdfx}
\hypersetup{hidelinks, bookmarksnumbered}
\usepackage{tocbibind}

\usepackage[top = 1in, right = 1in, bottom = 1in, left = 1.5in]{geometry}
\usepackage[doublespacing]{setspace}

\pagestyle{plain}

\usepackage{fontspec, unicode-math}
\setmainfont{XCharter}
\setmonofont{FiraMono}[Scale = 0.9]
\setmathfont{TeX Gyre Pagella Math}

\usepackage{minted}
\usemintedstyle{vs}
\setminted{fontsize = \small, baselinestretch = 1}
\setmintedinline{fontsize = \normalsize}

\begin{document}

\frontmatter

\begin{center}
\begin{singlespace}
\vspace*{0.5in}

\textbf{\uppercase{Yocto-CFA}}

\vspace*{1in}

by

Leandro Facchinetti

\vspace*{1.5in}

A dissertation submitted to Johns Hopkins University\\in conformity with the requirements for the degree of Doctor of Philosophy

\vspace*{0.5in}

Baltimore, Maryland

August 2020
\end{singlespace}
\end{center}

\thispagestyle{empty}
\clearpage

\chapter{Abstract}

% TODO

\paragraph{Primary Reader and Advisor:}

Dr.~Scott Fraser Smith.

\paragraph{Readers:}

Dr.~Zachary Eli Palmer and Dr.~Matthew Daniel Green.

\chapter{Acknowledgements}

% TODO

\tableofcontents
\listoftables
\listoffigures

\mainmatter

% TODO: Introduction

\chapter{Developing an Analyzer}

% TODO: An overview of the rest of the section

\section{The Analyzed Language: Yocto-JavaScript}

Our first decision when developing an analyzer is which language it should analyze. This is important because the analyzed language may affect the precision and the running time of the analyzer. For example, there is an analysis technique called \(k\)-CFA~\cite{k-cfa} that may be slower when applied to a language with higher-order functions than when applied to a language with objects (the algorithmic complexity of the former is exponential and the latter is polynomial)~\cite{m-cfa}.

In this dissertation we are interested in analysis techniques for higher-order functions. Fortunately, we have plenty of options, because most popular languages support higher-order functions: JavaScript, Java, Python, Ruby, and so forth. From all these languages, we choose JavaScript because it is the most widely used programming language~\cite{stack-overflow-developer-survey, jet-brains-developer-survey} and because it is understood by most people, even those who do not use it regularly, so it makes for a good language of discourse.

Unfortunately, JavaScript is a big language with many features besides higher-order functions, and if we tried to support all of them we would end up with an analyzer that is too complex. Instead, we choose to analyze only a subset of JavaScript features, namely the subset related to higher-order functions; we call this subset \emph{Yocto-JavaScript} (that is \(\mathrm{JavaScript} \times 10^{-24}\)).

\paragraph{Values in Yocto-JavaScript.}

JavaScript has many kinds of values: strings (for example, \mintinline{js}{"Leandro"}), numbers (for example, \mintinline{js}{29}), arrays (for example, \mintinline{js}{["Leandro", 29]}), objects (for example, \mintinline{js}!{ name: "Leandro", age: 29 }!), and so forth. Yocto-JavaScript has only one kind of value: functions. Yocto-JavaScript functions are written as something called \emph{arrow function expressions}~\cite{arrow-function-expressions}; for example, \mintinline{js}{x => x} is a function with one parameter (before the \mintinline{js}{=>}) called \mintinline{js}{x}, and a body (after the \mintinline{js}{=>}) which consists of just a reference to the variable \mintinline{js}{x}. Yocto-JavaScript functions return the result of computing their body, so our example function simply returns whatever argument it was passed. Yocto-JavaScript functions are limited to a single argument.

Because an Yocto-JavaScript function is a value, it may be the return value of another function, or it may be the argument in a call to another function. For example, the function \mintinline{js}{y => y} is the return value of the function \mintinline{js}{x => (y => y)} and it is the argument in the call to the function \mintinline{js}{x => x} in \mintinline{js}{(x => x)(y => y)}. This ability to behave as values is what characterize these functions \emph{higher-order}.

\paragraph{Operations in Yocto-JavaScript.}

JavaScript has many operations: numbers can be added together (for example, \mintinline{js}{29 + 1}), objects can have their properties accessed (for example, \mintinline{js}!{ name: "Leandro", age: 29 }.name!), and so forth. Yocto-JavaScript has only two operations: functions can be called and variables can be referenced.

The following is an example of these operations:

\begin{minted}{js}
(x => x)(y => y)
\end{minted}

In this program there is a function being called, \mintinline{js}{x => x}, and the argument being passed to it is \mintinline{js}{y => y} (functions are the only kind of value in Yocto-JavaScript, so the argument must itself be a function). As discussed above the \mintinline{js}{x => x} function simply returns whatever argument it was passed, so the result of the program is \mintinline{js}{y => y}.

The program above also includes examples of variable references: the \mintinline{js}{x} and \mintinline{js}{y} to the right of the respective \mintinline{js}{=>}.

We use parentheses to indicate the order in which the operations occur. For example, suppose \mintinline{js}{f}, \mintinline{js}{g}, and \mintinline{js}{h} are functions. There are at least two ways in which they can call one another:

\begin{enumerate}
\item \mintinline{js}{(f(g))(h)}: In this case \mintinline{js}{f(g)} happens first, and the result is a function that is called with \mintinline{js}{h}.

\item \mintinline{js}{f(g(h))}: In this case \mintinline{js}{g(h)} happens first, and the result is a function that is passed as argument to \mintinline{js}{f}.
\end{enumerate}

\paragraph{The Computational Power of Yocto-JavaScript.}

Yocto-JavaScript has very few features, which makes it an inconvenient language for programming, but it is ideal for our purpose of discussing the analysis of higher-order functions. Also, perhaps surprisingly, Yocto-JavaScript has the same computational power as JavaScript (and Java, Python, Ruby, and so forth), in the sense that any JavaScript program can be translated into an equivalent Yocto-JavaScript program.

To carry out this translation, we \emph{encode} the features used by the original JavaScript program in terms of features offered by Yocto-JavaScript. For example, consider a JavaScript program which includes a function with multiple parameters: \mintinline{js}{(x, y) => x}. We could translate this function into Yocto-JavaScript using a function with a single parameter that returns a function which accepts the second parameter: \mintinline{js}{x => (y => x)}. And we must translate every \emph{call} to that function in a similar fashion. Suppose the function was assigned to the variable \mintinline{js}{f}, then a call such as \mintinline{js}{f(a, b)} would be translated into \mintinline{js}{(f(a))(b)}.

In technical terms, we can say that Yocto-JavaScript is an incarnation of something called \emph{call-by-value \(\lambda\)-calculus}~\cite[§~6]{understanding-computation}), and we can describe its computational power by saying that it is \emph{Turing complete}~\cite[§~7]{understanding-computation}).

% TODO: Grammar.

\appendix

% TODO

\backmatter

\bibliographystyle{plain}
\bibliography{\jobname}

\chapter{Biographical Statement}

% TODO

\end{document}
